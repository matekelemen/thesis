% !TEX root = ../../thesis.tex
%______________________________________________________________________________
%
% SECTION
\section{Motivation}
\label{section:motivation}
%
%______________________________________________________________________________

The Finite Element Method (FEM) is a well established approach to obtaining approximate solutions to various problems in physics. Suitable for the discretization of a wide range of Partial Differential Equations (PDEs) and capable of dealing with a variety of geometries, it is often a convenient choice for getting detailed, accurate solution fields. However, it is not without drawbacks, some of which render it unsuitable for specific types of analyses. The most important disadvantages of the finite element method include its suboptimal convergence rate, lengthy and problematic mesh generation, and immense demand for computational resources.
These problems might not significantly affect moderately complex models, but become more important with increased accuracy requirements, complicated geometries, heterogeneous materials, or stringent resources. Numerous generalizations and modifications of the method have been explored that address one or several of the mentioned issues, but the ever-expanding number of applications still demand specialized improvements over the standard FEM.

Perhaps the most troublesome step of any FE analysis is mesh generation. While robust algorithms for creating acceptable 2D meshes on almost any geometry exist, the same cannot be said for 3D models in general. 3D meshes on complex geometries must often be adjusted by hand, a lengthy process that usually constitutes the bulk of the time spent performing the analysis. Additionally, mesh quality greatly influences the results, rendering consistent solutions for slight geometric modifications difficult to achieve. In some structured cases, macro elements that capture the geometry and material distribution inside can be used, but the lack of a general definition for such elements restricts their applicability. Furthermore, mesh generation is infeasible for some types of geometric representations, such as point clouds.
Embedded domain methods, and the Finite Cell Method (FCM) in particular, are shown to successfully tackle geometries that are otherwise difficult or impossible to mesh. Due to shifting the computational burden to numerical integration, they are easily automatable, completely eliminating the need for human supervision during the solution process. In addition, the method is well suited for h-p refinement, offering optimal convergence rates. However, these benefits come at the expense of efficiency for dynamic analyses, since the method inherently leads to dense mass matrices.

To solve dynamic PDEs, FE-based methods primarily rely on finite difference schemes that always involve inverting the mass matrix. The performance of any finite difference approach, but especially explicit schemes, greatly benefit from diagonal mass matrices, cutting their complexity down to a fraction of the original. Heuristic mass lumping procedures have a long and successful history with the linear FEM, but usually fail when applied to higher order methods. In contrast, the Spectral Element Method (SEM) features inherently diagonal consistent mass matrices for any order, offering both efficient time integration and high convergence rates at the expense of some restrictions.

The goal of this thesis is to explore approaches combining the advantages of the finite cell and spectral element methods, in pursuit of an efficient, accurate, and highly automatable means to solving dynamic PDEs on complex domains.