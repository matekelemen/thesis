% !TEX root = ../../thesis.tex
%______________________________________________________________________________
%
% SECTION
\section{Applications}
\label{section:applications}
%
%______________________________________________________________________________

Applications benefiting from such an approach address linear dynamic problems with high resolution requirements in time on complex domains, advanced geometric representation, or procedurally modified geometries.

High frequency wave propagation analysis in open cell foams, porous structures, or heterogeneous materials such as sandwich structures or other composites \cite{Joulaian2014} are all prime examples. Meshing such models is either impossible or immensely time consuming, resulting in unnecessarily numerous degrees of freedom that lead to high computational demand in both memory and processing power. Thus, a fictitious domain approach is evidently more appropriate. Furthermore, a high frequency analysis fundamentally restricts the size of time steps in order to correctly resolve the propagating waves, increasing the total number of time steps that constitute the bulk of the process' computational load. As a result, an efficient time integration scheme greatly reduces the total CPU time demanded by such an analysis.

Inverse problems are another class of candidates benefiting from efficient time integration and the lack of mesh generation. Adjoint-based dynamic inverse problems require repeated solutions on slightly modified models as part of an optimization loop. Since the optimization parameters are typically sensitive to changes in the solution, a consistently accurate method is paramount to achieving a convergent iteration, a task that automatic mesh generation is unsuitable for. A practical example is the optimal thermal load control of structures with heat limitations \cite{Alifanov1979}.

Another set of inverse problems are structural health monitoring methods, and full waveform inversion in particular. The goal of such processes is to reveal the internal geometry, often defects, of a model based on measurements on its surface. The objective of the optimization in this case is minimizing the difference between the computed solution and the measurements at all sample points by varying the internal geometry. The measurements correspond to the structure's response to high frequency excitation. The higher the frequency, the more detail internal defects can be detected in, but the smaller time steps must be in the simulation. Additionally, an appropriate representation of the structure's surface is not always available. For example, the geometry of heavily degraded buildings awaiting renovation is most conveniently captured using lidars, providing a point cloud as a surface representation \cite{Kudela2019}.

All mentioned applications can already be carried out using the FCM for instance, but their immense demand for computational resources could be reduced by successfully combining it with the SEM. The options of doing so are examined in the rest of this thesis, focusing on the acoustic wave equation as a test problem.

%______________________________________________________________________________
%
% SECTION
\section{Outline}
\label{section:outline}
%
%______________________________________________________________________________

The next chapter covers the relevant theoretical background of the physics and applied discretization methods. This includes the spatial finite element discretization of the acoustic wave equation followed by finite differences in time.
The core ideas of the FCM and SEM are explained afterwards, including their advantages and disadvantages. Lastly, the chapter features various approaches to evaluating the numerical solution to a dynamic system.

The details on combining the FCM and SEM are covered in the chapter on the Spectral Cell Method (SCM), summarizing the problems that arise from doing so and exploring possible approaches to addressing them.

Afterwards, results of models with varying complexity obtained using different methods are presented. The three featured setups involve a bar aligned with a Cartesian mesh, a bar rotated relative to the mesh, and an ellipsoid, all in three dimensions.

Finally, conclusions are drawn from the results and possible directions for further research are recommended.