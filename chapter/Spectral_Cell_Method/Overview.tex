% !TEX root = ../../thesis.tex
%______________________________________________________________________________
%
% SECTION
\section{Overview}
\label{section:overview}
%
%______________________________________________________________________________

What is the main idea of the SCM in a nutshell?
How does it differ from the previous methods?
What are the main challenges to be solved?

Expand:

The SCM is based on the FCM but uses basis functions and integration schemes
of the SEM in order to partially diagonalize the mass matrix. Cells that are
located entirely in the physical domain are handled identically as in the SEM.
However, integrating cut cells using methods from the FCM leads to non-zero
off-diagonal components in the mass matrix. The main challenge is to
robustly handle cut cells such that the resulting mass matrix remains diagonal.
Considered methods that attempt to tackle this problem are described in this
chapter.

The Spectral Cell Method (SCM) combines the main features of the finite cell and spectral element methods with the goal of creating an efficient but at the same time highly automatable algorithm for solving dynamic problems. The benefit is the ability to fully exploit the efficiency of explicit time integration schemes, even when working on arbitrarily complex geometries or heterogeneous materials. However, one has to keep in mind that the drawbacks and limitations of both approaches apply as well. Firstly, the types of PDEs are limited to linear ones because of the SEM's integration accuracy, mentioned in \ref{section:sem}. Even if that was not an issue, non-constant material parameters require the re-integration of structural matrices \ref{eq:structural_components} at every discrete time point, a task that far outweighs the complexity of a single time step. Furthermore, generating a boundary-conforming mesh should be difficult or the analysis must justify a fictitious domain approach, as the standard SEM would be more efficient otherwise. Lastly, the size of time steps $\Delta t$ should be limited by either the problem's physics or accuracy criteria because otherwise larger implicit time steps may prove to be more efficient than small explicit ones.

In a nutshell, the embedded domain concept of the FCM is used as a framework with Lagrange basis functions and Lobatto quadrature borrowed from the SEM. 