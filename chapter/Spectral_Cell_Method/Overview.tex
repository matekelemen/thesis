% !TEX root = ../../thesis.tex
%______________________________________________________________________________
%
% SECTION
\section{Overview}
\label{section:overview}
%
%______________________________________________________________________________

The Spectral Cell Method (SCM) combines the main features of the finite cell and spectral element methods with the goal of creating an efficient but at the same time highly automatable algorithm for solving dynamic problems. The benefit is the ability to fully exploit the efficiency of explicit time integration schemes, even when working on arbitrarily complex geometries or heterogeneous materials.

However, one has to keep in mind that the drawbacks and limitations of both approaches apply as well. Firstly, the types of PDEs are limited to linear ones because of the SEM's integration accuracy, mentioned in \ref{section:sem}. Even if that was not an issue, non-constant material parameters require the re-integration of structural matrices \ref{eq:structural_components} at every discrete time point, a task that far outweighs the complexity of a single time step. Furthermore, generating a boundary-conforming mesh should be difficult or the analysis must justify a fictitious domain approach, as the standard SEM would be more efficient otherwise. Lastly, the size of time steps $\Delta t$ should be limited by either the problem's physics or accuracy criteria because otherwise larger implicit time steps may prove to be more efficient than small explicit ones.

In a nutshell, the embedded domain concept of the FCM is used as a framework with Lagrange basis functions and Lobatto quadrature borrowed from the SEM. The stiffness matrix $\mathbf K$ and load vector $\mathbf f$ can be computed using other, more optimal integration schemes such as the Gauss-Legendre quadrature, for the same reasons detailed in \ref{section:sem}. As a result, the element mass matrices $\mathbf M^e$ of uncut cells are underintegrated, but inherently diagonal.

Since a Lagrange basis is used instead of an integrated Legendre one, the stiffness matrix must also be recomputed entirely when performing a p-refinement. However, this has little impact on the efficiency of the method because it only has to be reintegrated at the initial time step, the mass matrix and load vector must be recomputed either way, and the bulk of the computational load is expected to originate from time integration. A potentially more important issue is that element stiffness matrices are no longer optimally conditioned.