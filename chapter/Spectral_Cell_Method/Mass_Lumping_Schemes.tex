% !TEX root = ../../thesis.tex
%______________________________________________________________________________
%
% SECTION
\section{Mass Lumping Schemes}
\label{section:masslumpingschemes}
%
%______________________________________________________________________________

Mass lumping schemes are another group of candidates for providing diagonal mass
matrices for cut cells. They are a well established means of diagonalization
in the standard FEM, usually operating on the element or global matrix level
and eliminating off-diagonal components while lumping their contributions onto the
main diagonal. A number of such methods exist with the only universal requirement
being the preservation each element's total mass. The downside of mass lumping
techniques is that they are generally heuristic approaches and as such, their
accuracy may not improve with p-refinement in some cases. Moreover, their applicability is usually
restricted to certain types of elements or basis functions, and may provide negative
masses when these criteria are not met, leading to divergent solutions \cite{Duczek2019}. Three considered
methods are presented in the following sections.

As the SCM's viability hinges on the quality of diagonalized element mass matrices
of cut cells, choosing an appropriate lumping scheme is essential. To prevent underintegration, Gauss-Legendre-based adaptive quadrature is used for cut cells, since diagonalization through a choice of integration points is no longer possible.

%______________________________________________________________________________
%
% SUB-SECTION
\subsection{Row-Sum Lumping}
\label{section:rowsumlumping}
%
%______________________________________________________________________________

Row-sum lumping is one of the simplest approaches to diagonalization.
It is often used in the standard FEM and can even have some beneficial effects on the quality
of the solution for specific cases, such as adding artificial viscous behaviour to the system \cite{Hughes2000}.
The diagonal entries of an element mass matrix $M_{ii}^{e,RS}$ using the row-sum technique are computed as follows:

\begin{equation} \label{eq:rowsumlumping}
	\begin{array}{rl}
	M_{i}^{e,RS}
	
	&= \sum_k \int_{\Omega^e} N_i \rho N_k d\Omega \\[0.5em]
	&= \sum_k M_{ik}^e
		
	\end{array}
\end{equation}

Since the computation takes place on rows independently, the lumping can be executed
on the global mass matrix and can be trivially parallelized. However, note that
\ref{eq:rowsumlumping} holds only if the shape functions $N_i$ have a partition
of unity. Furthermore, this method may produce non-positive masses for high order ($p>2$)
basis functions \cite{Duczek2014}, which rules out its use in the SCM.

%______________________________________________________________________________
%
% SUB-SECTION
\subsection{Density Scaling}
\label{subsection:density_scaling_lumping}
%
%______________________________________________________________________________

Density scaling is described in \cite{Joulaian2014} and is unique to fictitious
domain methods, as the lumped matrix $M_{ij}^{e,DS}$ is given by the diagonal of the uncut cell's
consistent mass matrix $M_{ij}^{u,e}$, scaled to preserve the cut cell's mass $m_e$.

\begin{equation} \label{eq:densityscaling}
	M_{ii}^{e,DS} =
		\frac{m_e}{\sum_k M_{kk}^{u,e}} M_{ii}^{u,e}
\end{equation}

For the SCM, the resulting matrix in \ref{eq:densityscaling} is equivalent
to that of an uncut cell, whose density is scaled by the considered cell's mass ratio.

\begin{equation} \label{eq:scaleddensity}
	\rho_s = \frac{\int_{\Omega^e} \alpha \rho d\Omega}{\int_{\Omega^e} \rho d\Omega} \rho
\end{equation}

As a result, the geometric information on the boundary between the physical
and fictitious domains is lost in the process. Moreover, defining the density
outside the physical domain might not be straightforward for complex models, but
this issue is not addressed in the thesis as a constant density is assumed.

Compared to row-summing, this scheme imposes the additional overhead of having
to compute an extra mass matrix of an uncut cell, and it cannot be performed
on the assembled mass matrix. What it provides in return however, is physical
interpretability and a guaranteed positive-definite mass matrix.

%______________________________________________________________________________
%
% SUB-SECTION
\subsection{HRZ Lumping}
\label{section:hrzlumping}
%
%______________________________________________________________________________

HRZ lumping, also called diagonal scaling, discards off-diagonal components in the
element mass matrix and scales the diagonal such that the total mass is preserved \cite{Hinton1976}.

\begin{equation} \label{eq:hrzlumping}
	M_{ii}^{e,HRZ} =
		\frac{m^e}{\sum_k M_{kk}^e} M_{ii}^e
\end{equation}

Although this method guarantees the positive definiteness of the mass matrix, it
likely leads to reduced convergence rates \cite{Duczek2019}. Despite this shortcoming,
HRZ lumping is used in this thesis to lump the mass matrices of cut cells because
it provides positive diagonal components as opposed to row summing, it takes the
material distribution in the cell into account more accurately than density scaling, and
it does not impose significant overhead in the used software.

% -------------------------------------
\subsubsection*{Total mass of a cell}
\label{section:totalmassofacell}
% -------------------------------------

In order to compute the coefficient for scaling the diagonal components, the total
mass of the cell must be calculated first.

\begin{equation} \label{eq:celltotalmass}
	m^e = \int_{\Omega^e} \rho d\Omega
\end{equation}

Using basis functions that have partition of unity, \ref{eq:celltotalmass} can
be computed from the sum of all components in the element mass matrix.
\begin{equation} \label{eq:sumofmassmatrixcomponents}
	m^e = \sum_i \sum_j M_{ij}^e
\end{equation}


Functions in an ansatz space constructed from the tensor product of basis functions
retain partition of unity:

\begin{equation*}
\begin{array}{rll}
	\sum_k L_k(x) \equiv 1 & x \in [a,b] & a,b \in \rm I\!R \\
\end{array}
\end{equation*}

\begin{equation}
\begin{array}{rl}
	N_i &= N_{3^2r + 3s + t} = L_r L_s L_t \\
	\\
	\sum_i N_i
	&= \sum_r \sum_s \sum_t L_r L_s L_t	\\
	&= \sum_r
	\left(
		L_r \sum_s
		\left(
			L_s \sum_t L_t
		\right)
	\right) \\
	& \equiv 1
\end{array}
\end{equation}


As a result, the components of a cell's mass matrix add up to its total mass.
\begin{equation}
\begin{aligned}
	\sum_i \sum_j M_{ij}^e &= \sum_i \sum_j \int_{\Omega^e} \rho N_i N_j d\Omega \\
	&= \int_{\Omega^e} \rho \sum_i \sum_j N_i N_j d\Omega \\
	&= \int_{\Omega^e} \rho \sum_i
	\left(
		N_i \sum_j N_j
	\right)
	d\Omega \\
	&= \int_{\Omega^e} \rho d\Omega \\
	&= m^e
\end{aligned}
\end{equation}

This proves useful in the SCM, since lumping can be performed using only the consistent
element mass matrix obtained from adaptive integration, without having to separately
calculate the total mass.

\begin{equation}
	M_{ii}^{e,HRZ} =
	\frac{\sum_k \sum_l M_{kl}^e}{\sum_n M_{nn}^e} M_{ii}^e
\end{equation}