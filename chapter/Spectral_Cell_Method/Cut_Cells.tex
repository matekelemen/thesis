% !TEX root = ../../thesis.tex
%______________________________________________________________________________
%
% SECTION
\section{Cut Cells}
\label{section:cutcells}
%
%______________________________________________________________________________

What is the significance of cut cells?
What are the issues with them?
What are the options to handle them in the SCM?
Touch on badly cut cells too!

Cells that have points in both the fictitious and physical domains (cut cells)
need a modified integration scheme to resolve the boundary of the geometry. As
mentioned in \ref{section:fcm}, the most straightforward method is to use a
space partitioning scheme and perform the quadrature on the resulting subcells.
However, the additional integration points no longer coincide with roots of the
basis functions, which in turn lead to non-zero off-diagonal components in the
% TODO: add figure of a 1D cut cell with its additional integration points
mass matrix. As this defeats the purpose of the SCM, either the integration
scheme should be modified such that the number of integration points and their
positions are retained, or a post-integration lumping needs to be applied.