% !TEX root = ../../thesis.tex
%______________________________________________________________________________
%
% SECTION
\section{Error Computation}
\label{section:errorcomputation}
%
%______________________________________________________________________________

How should the quality of the solution be determined?
What are the difficulties of determining this quality?
What did I specifically compute for the example model?

What to touch on here:

Quantifying the quality of a solution to a dynamic system is not straightforward.
Directly comparing solution fields at each time step to a reference is often a
poor choice, as even small time delays can lead to huge errors.
% TODO: figure of a pair of identical, but shifted, functions
As later discussed in detail, the main concerns about the method in question
focus on the mass matrix. Inaccuracies in the mass matrix contribute to
the solution skewing over time and errors in the propagation velocity. Thus, it
makes sense to base the accuracy of a solution on the time of arrival of a wavelet
(and comparing the shape of the wavelet at different positions).
% TODO: figure about a travelling wave, skewing as it propagates

I can't describe the error calculation exactly unless I go through the toy model,
so that should be done here or somewhere before this point. It should probably be
put into its own section.

Describe the physics!
Describe the geometry!
Describe the boundaries!
Describe the load!
Describe the time domain!
Why was this model chosen in particular?

The chosen example model features the linear wave equation over a rectangular prism.
% TODO: 3D figure with sizes
All boundaries have zero-flux conditions and so are sources of reflection. The
time-dependent load is applied as a decaying volumetric source centered on one
of the prism ends, and is constant in the cross-sections.
% TODO: 1 or 2 figures uniquely defining the source
The magnitude of the source varies over time and is set to exactly one full sine wave,
which results in a sine half-wave displacement traveling down the length of the prism.
The symmetric nature of the load cancels reflections on the sides of the prism and
its excited end, but not on the other end. To filter displacement reflections coming
from the far end of the prism, the time domain is chosen as to only include the peak
displacement traveling through the length of the prism exactly once (based on the
analytical wave speed).

The displacement over time is sampled at the center of two different cross-sections.
The first cross-section should not be too close to the excited end, since part of the
displacement would become truncated in time. Similarly, the second cross-section must
not be near the far end, because reflections might skew the displacement wavelet.
As a result, solutions are sampled at [1/3] and [2/3] of the prism length.

What exactly is the error based on?
How is it computed?
Are there any caveats to this method?

The chosen measure of error relies on comparing the true wave speed (calculated from
the material properties) to that of the solution. Two main methods are used to compute
the wavelet's time of flight. One relies on finding the extremum of the traveling wavelet
while the other computes the centroid of the displacement's envelope.
% TODO: I should probably fit in a reference to Joulaian2014 somewhere around here.

The wavelet's extremum is computed as the extremum of a cubic spline interpolating
the displacements around the highest sampled absolute displacement. Computing the wave
speed solely relying on the wavelet peaks mostly neglects skewing. On the other hand,
it also means that the computed time of flight is unaffected by induced oscillations.
% TODO: figure of a spline interpolating the tip of a discretized sine half-wave

The other method, proposed by [Joulaian2014], is based on the envelope of the displacement,
computed as the absolute value of its Hilbert transform. The time of flight is then
calculated as the difference between said envelope's centroids. While this method takes
the wavelet's skewing into account, it is also sensitive induced oscillations and other
errors resulting from time integration, as displacements across all time steps are
included in the computation.
% TODO: figure of a signal, its envelope, and the centroid of that envelope

As these oscillations and errors tend to have much smaller magnitudes than the wavelet
itself, their significance can be reduced by squaring the envelope, or raising it to
a higher power, before finding its centroid. Note that as the exponent increases, the
centroid of the modified envelope approaches its maximum, thus the computed time of
flight converges on the result of the other method.
% TODO: figure showing how the centroid moves as the exponent increases