% !TEX root = ../../thesis.tex
%______________________________________________________________________________
%
% SECTION
\section{The Spectral Element Method}
\label{section:sem}
%
%______________________________________________________________________________

What problem does the SEM address?
What are the fundamental differences between SEM and standard FEM?
What are its advantages/disadvantages?
Where is it used?
Where should it not be used?

Main ideas to go through:

The primary feature of the SEM is that it provides an inherently diagonal mass matrix that
lends itself well to using explicit finite differences in time. By carefully choosing the
basis functions and the integration scheme, the mass matrix becomes diagonal without any
additional lumping required.

The underlying idea of the SEM is to use a set of Lagrange polynomials as basis functions,
interpolating all integration points.
% TODO: figure showing lagrange polynomials
This means that at each integration point, only one
basis function has a non-zero value, rendering all products of mixed basis functions zero
at that point. Since the product of basis function pairs appears as a coefficient in the
integrand of the mass matrix, all non-diagonal components vanish.

However, a couple of requirements constrain the choice of integration points. First of all,
in every FE-based method, each boundary of an element needs at least one basis function that does not vanish on it.
In the context of SEM, this requirement manifests as having exactly one non-zero basis function per
element boundary with its own set of integration points on it. In the local quadrature
space, this translates to integration points on both ends of the domain, which is a rare property
for quadrature rules to have. In fact, the only integration scheme (that I know of) that
satisfies this constraint is the GLL quadrature. Furthermore, the order of the basis functions
uniquely defines the number of integration points, since each basis function has exactly
one point at which it is non-zero, while having roots at all other ones. In summary, there is
no choice to be had in types of basis functions and once their polynomial order is set, so are the
integration points.

Another issue is the accuracy of the GLL quadrature. It exactly integrates polynomials up to degree
[2m-3] compared to [2m-1] for standard gaussian quadrature. Since the polynomial degree [p] of a
set of basis functions determines the size of the set [p+1], which in turn defines the number of
integration points [p+1], and considering that the integrand of the mass matrix involves
the product of pairs of basis functions, the highest polynomial degree in the integrand
(at least [2p] in the linear case) is greater than what GLL quadrature can exactly integrate [2p-1].
While some specific cases benefit from this underintegration, it is generally an additional
source of errors. Though this thesis focuses only on linear models, it should be noted that
a non-constant density function further increases the difference between the polynomial
degrees of the mass matrix's integrand, and that which can be exactly integrated in SEM.
% TODO: reference to a case where underintegration is beneficial

Naturally, derivatives of the basis functions will not yield the same property, so the
stiffness matrix will not be diagonal. In fact, there is no reason to choose GLL integration for
any terms other than that of the mass matrix.