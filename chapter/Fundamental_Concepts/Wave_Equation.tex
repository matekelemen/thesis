% !TEX root = ../../thesis.tex
%______________________________________________________________________________
%
% SECTION
\section{The Wave Equation}
\label{section:wave_equation}
%
%______________________________________________________________________________

Although the Finite Element Method (FEM) and its derivatives can be applied to a variety Partial Differential Equations (PDEs), the linear wave equation is chosen in this thesis as a simple model to demonstrate the discussed methods.
Its relation to full waveform inversion \cite{Zhang2013} and history with the Spectral Element Method (SEM) \cite{Maggio1994} makes it relevant for the topic, while its simplicity allows for focusing on the important aspects of the Spectral Cell Method (SCM).

The material is assumed to be homogeneous and isotropic, filling a geometry $\Omega \subset \rm I\!R^3$ with perfectly reflective boundaries $\Gamma = \partial \Omega$ modeled as homogeneous Neumann conditions. The time-dependent load $f(\mathbf x, t)=f_x(\mathbf x)f_t(t)$ is assumed to be separable and applied as a volumetric source.

\begin{equation} \label{eq:wave_equation}
	\begin{array}{rlr}
		\rho \cfrac{\partial^2 u(\mathbf x,t)}{\partial t^2} - E \Delta u(\mathbf x, t) &= f_x(\mathbf x) f_t(t)
		& \ \mathbf x \in \Omega \\[0.5em]
		\partial_n u(\mathbf x, t) &= 0
		& \mathbf x \in \Gamma \\[0.5em]
		u(\mathbf x, 0) &= u_0(\mathbf x)
		& \ \mathbf x \in \Omega \\[0.5em]
		\cfrac{\partial u(\mathbf x, 0)}{\partial t} &= v_0(\mathbf x)
		& \ \mathbf x \in \Omega \\
	\end{array}
\end{equation}

\begin{equation}
	\rho, E \in \rm I\!R^+
\end{equation}

where the displacement $u$ varies in space $\mathbf x \in \Omega$ and time $t \in [0,t_{max}]$, the material is modeled with a constant density $\rho$ and Young's modulus $E$, and $\partial_n$ representing the derivative normal to the geometry's boundary $\Gamma$. The initial state of the system at $t=0$ is defined by its displacement $u_0(\mathbf x)$ and velocity $v_0(\mathbf x)$ field.
The PDE \ref{eq:wave_equation} is discretized by finite elements in space, and finite differences in time. Since the theory and details of both discretizations have extensive literature, only a brief overview relevant to the covered topics is provided here. A more detailed derivation can be found in \cite{Larson2013} for example.

%______________________________________________________________________________
% SUB-SECTION
\subsection{Spatial Discretization}
\label{subsection:wave_equation_spatial_discretization}
%______________________________________________________________________________

The first step to obtaining a weak form is to multiply \ref{eq:wave_equation} with a test function $v(\mathbf x)$ from an appropriate ansatz space $V$, and integrate over the spatial domain.

\begin{equation}
	\int_{\Omega} v(\mathbf x) \rho \cfrac{\partial^2 u(\mathbf x, t)}{\partial t^2} \ d \mathbf x
	-
	\int_{\Omega} v(\mathbf x) E \Delta u(\mathbf x, t) \ d \mathbf x
	=
	f_t(t) \int_{\Omega} v(\mathbf x) f_x(\mathbf x) \ d \mathbf x
\end{equation}

Integrating by parts and substituting the homogeneous Neumann boundary condition yields the weak form of the wave equation.

\begin{equation} \label{eq:wave_equation_weak_form}
	\int_{\Omega} v \rho \cfrac{\partial^2 u}{\partial t^2} \ d \mathbf x
	+
	\int_{\Omega} (\nabla v)^T E \nabla u \ d \mathbf x
	=
	f_t \int_{\Omega} v f_x \ d \mathbf x
\end{equation}

The spatial discretization is carried out by approximating the solution field and test function with linear combinations of ansatz functions $N_i(\mathbf x) \in V_h \subset V$.

\begin{equation}
	\begin{array}{rl}
		u(\mathbf x, t) &\cong \ \sum_i N_i(\mathbf x) \hat u_i(t) \\
		v(\mathbf x) &\cong \ \sum_i N_i(\mathbf x) \hat v_i \\
	\end{array}
\end{equation}

\begin{equation} \label{eq:wave_equation_spatially_discretized_test}
	\int_{\Omega} \sum_i N_i \hat v_i \ \rho \sum_j N_j \cfrac{\partial^2 \hat u_j}{\partial t^2} \ d \mathbf x
	+
	\int_{\Omega} \sum_i \left( \nabla N_i \hat v_i \right)^T E \sum_j \nabla N_j \hat u_j \ d \mathbf x
	=
	\int_{\Omega} f \sum_i N_i \hat v_i d \mathbf x
\end{equation}

Since the weak form \ref{eq:wave_equation_weak_form} must hold for any $v \in V_h$, equation \ref{eq:wave_equation_spatially_discretized_test} has to hold for all combinations of $\hat v_i$ as well, leading to the spatially discretized form of the wave equation:

\begin{equation} \label{eq:wave_equation_spatially_discretized}
	\int_{\Omega} N_i \rho N_j d \mathbf x \ \cfrac{\partial^2 \hat u_j}{\partial t^2}
	+
	\int_{\Omega} \left( \nabla N_i \ \right)^T E \ \nabla N_j d \mathbf x \ \hat u_j
	=
	f_t \int_{\Omega} f_x N_i d \mathbf x
\end{equation}

The mass matrix $\mathbf M$, stiffness matrix $\mathbf K$ and load vector $\mathbf f$ appear in \ref{eq:wave_equation_spatially_discretized}.

\begin{equation} \label{eq:wave_equation_structural_components}
M_{ij} \cfrac{\partial^2 \hat u_j}{\partial t^2}
+
K_{ij} \hat u_j
=
f_t f_i
\end{equation}

\begin{equation} \label{eq:structural_components}
	\begin{array}{rl}
		M_{ij} &= \ \int_{\Omega} N_i \rho N_j dx \\
		K_{ij} &= \ \int_{\Omega} \left( \nabla N_i \right)^T E \nabla N_j dx \\
		f_{i}  &= \ \int_{\Omega} f_x N_i dx \\
	\end{array}
\end{equation}

%______________________________________________________________________________
% SUB-SECTION
\subsection{Temporal Discretization}
\label{subsection:wave_equation_temporal_discretization}
%______________________________________________________________________________

The result of the spatially discretized wave equation \ref{eq:wave_equation_structural_components} is a second order system of Ordinary Differential Equations (ODEs), identical in form for many linear undamped models in structural mechanics and other fields. Although variational time discretization methods with appealing properties have been explored \cite{Zhao2014} and applied to the wave equation \cite{Kocher2014}, finite difference schemes dominate the literature most likely due to their simplicity, long history, and efficiency in both computational load and memory requirements.

Two main categories of finite difference schemes exist with distinct properties. To allow a more general discussion on them, \ref{eq:wave_equation_structural_components} is transformed into a first order system of the form $\dot{\mathbf y}(t) = \mathbf g(t, \mathbf y(t))$

\begin{equation} \label{eq:wave_equation_first_order_form}
	\dot{\mathbf y}(t) := 
	\begin{bmatrix}
		\dot{\hat{\mathbf u}}(t) \\
		\ddot{\hat{\mathbf u}}(t) \\
	\end{bmatrix}
	=
	\begin{bmatrix}
		0 & \mathbf I \\
		- \mathbf M^{-1} \mathbf K & 0 \\
	\end{bmatrix}
	\begin{bmatrix}
		\hat{\mathbf u}(t) \\
		\dot{\hat{\mathbf u}}(t) \\
	\end{bmatrix}
	+
	\begin{bmatrix}
		0 \\
		f_t \mathbf M^{-1} \mathbf f
	\end{bmatrix}
	=
	\mathbf g(t, \mathbf y(t))
\end{equation}

where $\dot{\hat{\mathbf u}} = \cfrac{d \hat{\mathbf u}}{d t}$ denotes time differentiation and the extended state is defined as $\mathbf y = [\hat{\mathbf u} \ \dot{\hat{\mathbf u}}]^T$. A generic finite difference scheme marches at discrete points in time $t_k$ to approximate the system's state $\mathbf y_k = \mathbf y(t_k)$ using \ref{eq:wave_equation_first_order_form}.

\begin{equation} \label{eq:generic_finite_differences}
	\mathbf y_{k+1} \approx \mathbf h^i(\mathbf y_{k+1}) + \mathbf h^e(\mathbf y_k, \mathbf y_{k-1}, ..., \mathbf y_0)
\end{equation}

For implicit methods, $\mathbf h^i \neq 0$ meaning that they include the yet unknown state $\mathbf y_{k+1}$, requiring the solution of an algebraic equation system at each time step. These methods are unconditionally stable and may prove to be a suitable choice for problems where large time steps are acceptable.

On the other hand, explicit methods only use previously computed states ($\mathbf h^i = 0$ and $\mathbf h^e \neq 0$), requiring the evaluation of the right hand side in \ref{eq:wave_equation_first_order_form} at earlier time points, and provides an explicit expression for the unknown state. If the mass matrix can be inverted cheaply, explicit time steps can be performed far more efficiently than implicit ones, at the cost of stability. To be convergent, explicit finite difference schemes must satisfy the Courant-Friedrichs-Lewy (CFL) condition \cite{Courant1967}, which defines an upper bound for the time step size relative to the spatial discretization. This requirement can defeat the purpose of using explicit methods in some cases as the computational load per step size may exceed that of implicit ones. As mentioned in \ref{chapter:introduction} however, the assumption in this thesis is that the physics of the problem already demand small time steps, justifying the use of explicit finite differences.

A popular method for the temporal discretization of \ref{eq:wave_equation_first_order_form} is the Newmark-beta scheme \cite{Newmark1959} that assigns tunable weights to the accelerations separately for the approximation of the velocity and displacement. Depending on the choice of parameters $\beta$ and $\gamma$, it can either be implicit or explicit.

\begin{equation}
	\begin{array}{rl}
		\hat{\mathbf{u}}_{k+1} &= \ \hat{\mathbf{u}}_k + \Delta t \dot{\hat{\mathbf{u}}}_k + \Delta t^2
		\cfrac{(1-2 \beta) \ddot{\hat{\mathbf{u}}}_k + 2 \beta \ddot{\hat{\mathbf{u}}}_{k+1}}{2} \\[0.5em]
		\dot{\hat{\mathbf{u}}}_{k+1} &= \ \dot{\hat{\mathbf{u}}}_k + \Delta t \left(
		(1 - \gamma) \ddot{\hat{\mathbf{u}}}_k + \gamma \ddot{\hat{\mathbf{u}}}_{k+1}
		\right) \\
	\end{array}
\end{equation}

The second order Central Difference Method (CDM) is obtained by setting the parameters $\beta=0$ and $\gamma=\frac{1}{2}$.

\begin{equation} \label{eq:central_differences_raw}
	\begin{array}{rl}
		\hat{\mathbf u}_{k+1} &= \hat{\mathbf u}_k + \Delta t \dot{\hat{\mathbf u}}_k + \cfrac{\Delta t^2}{2} \ddot{\hat{\mathbf u}}_k \\
		\dot{\hat{\mathbf u}}_{k+1} &= \dot{\hat{\mathbf u}}_k + \cfrac{\Delta t}{2}( \ddot{\hat{\mathbf u}}_k + \ddot{\hat{\mathbf u}}_{k+1} ) \\
	\end{array}
\end{equation}

To show that \ref{eq:central_differences_raw} is explicit, the first derivatives can be eliminated:

\begin{equation} \label{eq:central_differences}
	\hat{\mathbf u}_{k+1} = 2\hat{\mathbf u}_k - \hat{\mathbf u}_{k-1} + \Delta t^2 \ddot{\hat{\mathbf u}}_k
\end{equation}

Substituting \ref{eq:wave_equation_first_order_form} into \ref{eq:central_differences} leads to a fully discretized form of the wave equation.

\begin{equation} \label{eq:wave_equation_fully_discretized}
	\hat{\mathbf u}_{k+1} = 2\hat{\mathbf u}_k - \hat{\mathbf u}_{k-1} + \Delta t^2 \mathbf M^{-1}(f_t \mathbf f_k - \mathbf K \hat{\mathbf u}_k)
\end{equation}

Note that \ref{eq:wave_equation_fully_discretized} requires the inversion of the mass matrix $\mathbf M$, which is a common property of all explicit methods. Most often, $\mathbf M$ is factorized before the initial time step instead of directly computing its inverse. However, the ideal case would be if the mass matrix were diagonal, eliminating the need of forward/backward substitutions of the factorized $\mathbf M$ and reducing the number of matrix-vector operations. Exploring different approaches to achieve this in combination with the finite cell method is the main goal of this thesis.