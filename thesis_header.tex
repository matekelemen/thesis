% !TEX root =  thesis.tex
\documentclass[11pt,twoside,a4paper]{book}
\usepackage[utf8]{inputenc}
\usepackage{amsmath}
\usepackage{amssymb}
%f�r das Einbinden von Bildern und Tabellen
\usepackage{booktabs} 		%f�r tolle tabellen
\usepackage{threeparttable}
\usepackage{longtable}
\usepackage[normal,small,bf]{caption} %erm�glicht mehrzeilige Bildunterschriften use [hang] f�r einger�ckte Bildunterschrift
\usepackage{psfrag}
%f�r die Bibliography
\usepackage{natbib}		% more flexibility with citations
\usepackage{ifpdf}
\usepackage{footnote}
%Verwendung von Farben
\usepackage{xcolor}
%bedruckten Bereich der Seite anpassen
\usepackage[left=3.0cm, right=2.5cm, top=2.5cm,bottom=2.0cm,includeheadfoot]{geometry}
% Formatierung Kapitelueberschriften
\usepackage{titlesec}
\usepackage{framed}
\usepackage{array}
%\titleformat{?�berschriftenklasse?}[Absatzformatierung?]{?Textformatierung?} {?Nummerierung?}{?Abstand zwischen Nummerierung und �berschriftentext?}{?Code vor der �berschrift?}[?Code nach der �berschrift?]
%\titleformat{\chapter}[hang]{\huge\bfseries}{\thechapter\quad}{0pt}{}
%\titlespacing{?�berschriftenklasse?}{?Linker Einzug?}{?Platz oberhalb?}{?Platz unterhalb?}[?rechter Einzug?]
%\titlespacing{\chapter}{0pt}{0em}{6pt}
%----------------------------------------------------------------------------------------------------------------------------------------------
%Stil der Seite
\usepackage{fancyhdr}
\fancypagestyle{plain}{%
	\fancyhf{} % clear all header and footer fields
	\fancyhead[EL,OR]{\slshape \thepage} % except the center
	\fancyhead[ER]{\slshape \leftmark}
	\fancyhead[OL]{\slshape \rightmark}
}
\renewcommand{\chaptermark}[1]{\markboth{\thechapter. #1}{}}
\renewcommand{\sectionmark}[1]{\markright{\thesection. #1}{}}
%
\pagestyle{empty}%f�r die Titelseiten zumindest, am Ende der Titelseite umschalten auf \pagestyle{fancyplain}
\pagenumbering{Roman}%f�r die Titelseiten zumindest, am Ende der Titelseite umschalten auf \pagenumbering{arabic}
%------------------------je nach Kommando (pdflatex / latex) jeweilige Paketeinbindung-----------------------------
\definecolor{linkblue}{rgb}{0,0.1,0.6}
\definecolor{citegreen}{rgb}{0,0.25,0.15}%{0.1,0.5,0.4}%{0.125,0.6,0.5}
\definecolor{linkred}{rgb}{0.8,0,0.005}%{0.6,0,0.1}
\definecolor{mailviolet}{rgb}{0.3,0,0.35}%{0.6,0,0.1}
\definecolor{tumblue}{rgb}{0,0.396,0.741}
\ifpdf %pdflatex
    \usepackage[pdftex]{graphicx}
    \pdfcompresslevel=1
    \pdfimageresolution=300
    \DeclareGraphicsExtensions{.png}
    \graphicspath{{./pictures/}}
    \usepackage[hyperindex,pdftex,colorlinks=true,linkcolor=tumblue,citecolor=citegreen,urlcolor=mailviolet,filecolor=linkred]{hyperref}
\else %latex && dvips
    \usepackage[dvips]{graphicx}
    \DeclareGraphicsExtensions{.eps}%.bmp,.tif,.tiff,.tga}
    \graphicspath{{./pictures/}}
  \usepackage[hyperindex,pdfmark,dvips,colorlinks=true,linkcolor=tumblue,citecolor=citegreen,urlcolor=mailviolet,filecolor=linkred]{hyperref}
\fi
%Zeilenabst�nde regulieren mit \singlespacing, \onehalfspacing, \doublespacing
\usepackage{subcaption} %{subfigure} alt
\usepackage{setspace}
\onehalfspacing
\parindent 0pt
%Schriftart Sans Serif
%\renewcommand*\familydefault{\sfdefault}

\usepackage{pgfplots}
